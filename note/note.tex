%        File: notes.tex
%     Created: Thu Nov 07 11:00 AM 2013 C
% Last Change: Thu Nov 07 11:00 AM 2013 C
%
\documentclass[a4paper]{article}

% vim:foldmethod=marker
% \usepackage{lmodern}
% \usepackage[sc]{mathpazo}
% \usepackage{tgpagella}
\usepackage[T1]{fontenc}
\usepackage[utf8]{inputenc}
\usepackage[american]{babel}

% include bibliography in TOC
% http://en.wikibooks.org/wiki/LaTeX/Bibliography_Management#Using_tocbibind
\usepackage[nottoc]{tocbibind}

\usepackage{graphicx}
% Allow spaces in graphics files
\usepackage{grffile}

\usepackage{subfig}
\usepackage[labelfont=bf]{caption}

\usepackage{lipsum}
\usepackage{multirow}
\usepackage{booktabs}

\usepackage{enumitem}
\usepackage{footmisc}

\usepackage{calc}

\usepackage{amsmath}
\usepackage{amssymb}
\usepackage{maybemath}


% To fix
% Package biblatex Warning: 'babel' detected but 'csquotes' missing.
% (biblatex)                Loading 'csquotes' recommended.
\usepackage{csquotes}
\usepackage[colorlinks=false]{hyperref}
\usepackage[style=numeric-comp,
  firstinits=true,
  maxbibnames=4,
  maxcitenames=2,
  minnames=1,
  hyperref=true,
  sorting=none,
  sortcites=true,
  backend=biber,
  eprint=true,
  doi=true,
  url=true,
  isbn=false
]{biblatex}
\bibliography{/home/vseggern/papers2/bibliography.bib}
% make article titles sentence cased
% see
% <http://tex.stackexchange.com/questions/22980/sentence-case-for-titles-in-biblatex>
\DeclareFieldFormat{sentencecase}{\MakeSentenceCase*{#1}}
\renewbibmacro{title}{%
    \ifthenelse{\iffieldundef{title}\AND\iffieldundef{subtitle}}{}
        {\ifthenelse{\ifentrytype{article}\OR\ifentrytype{inbook}%
            \OR\ifentrytype{incollection}\OR\ifentrytype{inproceedings}}
            {\printtext[title]{%
                \printfield[sentencecase]{title}%
                \setunit{\subtitlepunct}%
                \printfield[sentencecase]{subtitle}}%
                \newunit}%
            {\printtext[title]{%
                \printfield[titlecase]{title}%
                \setunit{\subtitlepunct}%
                \printfield[titlecase]{subtitle}}%
                \newunit}}%
    \printfield{titleaddon}}
% http://tex.stackexchange.com/questions/10682/suppress-in-biblatex
\renewbibmacro{in:}{%
  \ifentrytype{article}{}{%
    \printtext{\bibstring{in}\intitlepunct}}}
% Do not set language field
% see
% <http://tex.stackexchange.com/questions/32930/is-it-possible-to-suppress-a-specific-field-from-bibtex-bbl-in-biblatex>
\AtEveryBibitem{\clearlist{language}} % clears language

% See http://latex-alive.tumblr.com/post/56318007101
% NOTE: Use `~' here because I nobreakbefore inserts a non-breaking
% space only if a space is present before \cite{...}
\def\nobreakbefore{%
  \relax\ifvmode\else
    \ifhmode
      \ifdim\lastskip > 0pt\relax
        \unskip\nobreakspace
      \fi
    \fi
  \fi
}
\def\alwaysnobreakbefore{%
  \unskip\nobreakspace
}
\let\origcite\cite
\renewcommand{\cite}{\alwaysnobreakbefore\origcite}

\usepackage{xspace}
\usepackage{relsize}

\usepackage{lscape}
\usepackage{afterpage}

\usepackage{siunitx}
\sisetup{table-number-alignment=center,group-separator=none,range-phrase=dash,separate-uncertainty=true}
% \newunit{\ADU}{ADU}
% \newunit{\ec}{\Pelectron^{-}}
% % TODO 
% \newunit{\photon}{\mbox{$\gamma$}}
% \newunit{\pixel}{px}
% \newunit{\parsec}{pc}
% \newunit{\year}{yr}
% \newunit{\erg}{erg}
% \newunit{\ppm}{ppm}
% \newunit{\milliarcsecond}{mas}
% \newunit{\arcsecond}{as}

\newcommand*\SIsym[2]{\SI[per-mode = symbol]{#1}{#2}}
\newcommand*\sisym[1]{\si[per-mode = symbol]{#1}}

\usepackage{xcolor}

\usepackage{tikz}
\usetikzlibrary{calc,positioning,scopes,arrows,decorations.markings,decorations.pathmorphing,shapes.geometric}
\tikzset{axis line/.style={->,thick}}
\tikzset{axis tick/.style={thick}}
\tikzset{orientation line/.style={dashed}}
\tikzset{function line/.style={very thick}}

\tikzset{filled charge bucket/.style={lightgray}}
\tikzset{electric connection/.style={thick}}
\tikzset{electric plug/.style={thick,fill=white}}
 \edef\defaultpgflinewidth{\the\pgflinewidth}

%% TODO boxes {{{
\definecolor{todoboxcolor}{rgb}{1.0,0.6,0.6}
\newcommand{\TODO}[1]{\par\medskip%
\colorbox{todoboxcolor}{TODO: \parbox[t]{8.5cm}{#1}}%
\medskip\newline}

\definecolor{todoneboxcolor}{rgb}{1.0,0.9,0.9}
\newcommand{\TODONE}[1]{\par\medskip \noindent%
\colorbox{todoneboxcolor}{TODONE?: \parbox[t]{8.5cm}{#1}}%
\medskip\newline}

\newcommand\TODOI[1]{\colorbox{todoboxcolor}{\textsc{\smaller Todo:}\textbf{#1}}}

\definecolor{tocitecolor}{rgb}{0.8,0.2,0.2}
\newcommand\TOCITE[1]{\textbf{\color{tocitecolor}[#1]}}
%% }}}


\newcommand*\disptextwidth{\\\rule{0.1pt}{10pt}\rule[3pt]{\textwidth-0.2pt}{0.1pt}\rule{0.1pt}{10pt}}

%% References {{{
\newcommand\Eq[1]{Eq.~\eqref{#1}\xspace}
\newcommand\Eqs[1]{Eqs.~\eqref{#1}\xspace}
\newcommand\Fig[2][]{Fig.~\ref{#2}#1\xspace}
\newcommand\Figs[2][]{Figs.~\ref{#2}#1\xspace}
\newcommand\Tab[1]{Tab.~\ref{#1}\xspace}
\newcommand\Tabs[1]{Tabs.~\ref{#1}\xspace}
\newcommand{\Sec}[1]{Sec.~\ref{#1}\xspace}
\newcommand{\Secs}[1]{Secs.~\ref{#1}\xspace}
\newcommand{\Chap}[1]{Chap.~\ref{#1}\xspace}
\newcommand{\Chaps}[1]{Chaps.~\ref{#1}\xspace}
\newcommand*{\App}[1]{App.~\ref{#1}\xspace}
%% }}}

%% Formula abbreviations {{{{{
\newcommand\mean[1]{\ensuremath{\langle #1\rangle}\xspace}
\newcommand\ofOrder[1]{\ensuremath{\mathcal{O}(#1)}\xspace}
\DeclareMathOperator{\var}{var}
\DeclareMathOperator{\rms}{rms}
\DeclareMathOperator{\const}{const}
\DeclareMathOperator{\median}{median}
\DeclareMathOperator*{\argmax}{arg\,max}

\newcommand\myvec[1]{\ensuremath{\mathbf{#1}}\xspace}

% punctuation at the end of formulas
\newcommand\fcomma{\;,}
\newcommand\fstop{\;.}

\newcommand\diff{\mathrm{d}}

\newcommand\poi{\ensuremath{\mathrm{Poi}}\xspace}
\newcommand\gau{\ensuremath{\mathcal{N}}\xspace}

\newcommand\CL{\ensuremath{\mathrm{CL}}\xspace}

%% }}}}}

\newcommand\xparagraph[1]{\textbf{#1:}\xspace}
% \newcommand\myem[1]{\textbf{#1}\xspace}
\newcommand\ie{i.e.~}
\newcommand\eg{e.g.~}

\newcommand\NTH[2]{#1$^\mathrm{#2}$\xspace}
%% Names of experiments
%% {{{
%% }}}


\addbibresource{conflim.bib}

\newcommand*\Hz{\maybebm{\mathrm{H}_0}\xspace}
\newcommand*\Prob{\mathrm{Pr}}
\newcommand*\PG{\maybebm{P_\mathrm{G}}\xspace}
\newcommand*\Pchi{\maybebm{P_{\chi^2}}\xspace}

\title{A Semi-analytic Solution for Unified Confidence Intervals Based
on Critical Likelihood-ratio Values}
\author{Jan Eike von Seggern}
\begin{document}
\maketitle

\section{Introduction}

\textcite{Feldman1998} proposed to construct Frequentist confidence
intervals
using an ordering scheme based on likelihood ratios. This method is
known as the ``Unified Method''. It allows to avoid so-called
flip-flopping between quoting one- and two-sided intervals depending on
the measured value, which is a source of significant undercoverage.

In their original paper, \citeauthor{Feldman1998} use the classical
Neyman prescription to construct confidence belts, which are
subsequently used to derive confidence intervals based on a measured
value. However, the construction confidence intervals is tightly
connected to statistical hypothesis
testing\cite{wiki:confidenceinterval,Cox1974}: The confidence intervals
can be constructed by including all values for which a test statistic is
above a critical value\cite{Sen2009}.

In this note, we will present a semi-analytical method to construct
unified confidence intervals for the case of estimating the expectation
value, $\mu$, for a Gaussian-distributed observable with known variance,
$\sigma^2$, and constraint $\mu \geq 0$. In contrast to the classical
Neyman procedure, this method does not rely on constructing a confidence
belt.

\section{Confidence Intervals and Hypothesis Testing}

As mentioned above, the construction of confidence intervals is tightly
connected to hypothesis testing: Consider estimating the parameter
$\theta$ of a single-parameter distribution, \eg the expectation value
of a Gaussian distribution. The confidence interval (or region) for
$\theta$ with confidence level $\CL = 1-\alpha$ can be constructed to be
the set of values $\{\theta\}$ for which the null hypothesis
%
\begin{equation*}
  \Hz: \theta = \theta_\mathrm{true}
\end{equation*}
%
is not rejected with significance $\alpha$ for a measured value of $x$.

Let $t(x)$ be the test statistic used with $t=0$ if \Hz is true. Then,
\Hz is rejected if
%
\begin{equation*}
  |t(x)| \geq c_\alpha
  \fcomma
\end{equation*}
%
where the critical value $c_\alpha$ is defined by
%
\begin{equation*}
  \Prob\left( |t| \geq c_\alpha; \theta\right) = \alpha
  \fcomma
\end{equation*}
%
\ie the probability to measure a sample $x$ with $|t(x)| > c_\alpha$ is
$\alpha$ if the parameter of the distribution has the value $\theta$.
One should note that the critical value may be a function of the
parameter: $c_\alpha=c_\alpha(\theta)$.

In order to construct unified confidence intervals, the likelihood ratio
%
\begin{equation*}
  \Lambda(\theta) = \frac{L(\theta; x)}{L(\hat{\theta}(x); x)}
\end{equation*}
%
% or for convenience its logarithm
% %
% \begin{equation*}
  % \lambda = -2 \ln \Lambda
% \end{equation*}
% %
is used as test statistic, where $L$ is the likelihood function and
$\hat{\theta}(x)$ maximizes $L$ for given $x$ obeying constraints (\eg
$\theta \geq 0$).
Hence, $\Lambda$ takes up values in
$[0,1]$. If the null hypothesis,
$\hat{\theta}(x) = \theta_\mathrm{true}$, is true, we find $\Lambda=1$
slightly different to the definition of the test statistic $t$ above.
The critical value is therefore defined by the $\alpha$-quantile of the
distribution of the likelihood ratio,
%
\begin{equation}
  \label{eq:likelihood:ratio:critical:value}
  \Prob(\Lambda < c_\alpha; \theta) = \alpha
\end{equation}
%
and the confidence interval consists of all values $\theta$ for which
%
\begin{equation*}
  \Lambda(\theta; x) \geq c_\alpha(\theta)
  \fstop
\end{equation*}
%

\section{Likelihood-ratio Distribution for Constrained Gaussian}

As seen above, it is necessary to calculate the $\alpha$-quantile of the
likelihood-ratio distribution. Therefore, it is necessary to know the
distribution of the likelihood-ratio, which will be derived in this
section for the case of the mean of a Gaussian that is constrained to
the positive domain. For a single measurement, $x$, the
maximum-likelihood estimator is given by\cite{Feldman1998}
%
\begin{equation*}
  \hat\mu(x) = \max(0, x)
  \fstop
\end{equation*}
%
Hence, the likelihood ratio is given by
%
\begin{equation*}
  \Lambda(\mu; x) = 
  \begin{cases}
    \displaystyle
    \exp\left[ - \frac{(x-\mu)^2}{2\sigma^2} \right] & \text{if } x \geq 0
    \fcomma
    \\
    \displaystyle
    \exp\left[ \frac{\mu}{\sigma^2} \left( t - \frac{\mu}{2} \right) \right] & \text{otherwise}
    \fstop
  \end{cases}
\end{equation*}
%

For convenience, we will use $\lambda=-2\ln\Lambda$ instead,
%
\begin{equation}
  \label{eq:log:likelihood:ratio}
  \lambda(\mu; x) = 
  \begin{cases}
    \displaystyle
     \frac{(x-\mu)^2}{\sigma^2} & \text{if } x \geq 0
    \fcomma
    \\
    \displaystyle
    \frac{2\mu}{\sigma^2} \left( \frac{\mu}{2} - x \right) & \text{if } x < 0
    \fcomma
  \end{cases}
\end{equation}
%
which takes up values from $[0, \infty)$.
According to \Eq{eq:likelihood:ratio:critical:value} and because the
logarithm is strictly increasing,
the critical $\lambda$-value, $\zeta_\alpha = -2\ln c_\alpha$, is given by
%
\begin{equation*}
  \Prob(\Lambda < c_\alpha; \mu) = 
  \Prob(\lambda > \zeta_\alpha; \mu) = \alpha
  \fstop
\end{equation*}
%

The cumulative distribution function (CDF) for $\lambda$ is given by
\begin{align*}
  P_\lambda(l) & = \Prob(\lambda < l; \mu) \\
  & =
  \underbrace{\Prob(\lambda < l \wedge x \geq 0; \mu)}_{P_+(l)}
  +
  \underbrace{\Prob(\lambda < l \wedge x < 0; \mu)}_{=P_-(l)}
  \fcomma
\end{align*}
where the contributions $P_\pm$ correspond to the two cases of
\Eq{eq:log:likelihood:ratio} and $\mu$ is mentioned explicitly to
remind that the probabilities are evaluated under the assumption that
the expectation value of the $x$-distribution is $\mu$.

\paragraph{$\maybebm{P_+}$:}
The probability contribution $P_+$ can be rewritten using conditional
probabilities,
%
\begin{align}
  P_+(l)
  & = \Prob(\lambda < l \wedge x \geq 0; \mu)
  \nonumber
  \\
  & = \Prob(\lambda < l | x \geq 0) \cdot \Prob(x \geq 0; \mu)
  \label{eq:p:plus:conditional}
  \fcomma
\end{align}
%
where $\Prob(x \geq 0; \mu)$ is given by the Gaussian CDF, \PG, by
%
\begin{equation}
  \label{eq:p:plus:prior}
  \Prob(x \geq 0; \mu) = 1 - \PG(0; \mu, \sigma)
  \fstop
\end{equation}
%

Because $x$ is Gaussian-distributed, 
$\lambda = [(x-\mu)/\sigma]^2$ follows in principle a
chi-squared distribution with one degree of freedom. However, due to the
constraint $x \geq 0$, we find
%
\begin{equation*}
  \frac{t-\mu}{\sigma} > -\frac{\mu}{\sigma}
\end{equation*}
%
and the parabola $\lambda=[(x-\mu)/\sigma]^2$ is realized only once
for values above $(\mu/\sigma)^2$ but twice for values below this
bound. Hence, the chi-squared CDF, \Pchi, has to be re-weighted
in order to calculate $P_+$. The factor for re-weighting is given by
\begin{equation*}
  \overbrace{2 \cdot \Pchi[(\mu/\sigma)^2]}^{\text{contribution
  for} \lambda < (\mu/\sigma)^2} +
  \overbrace{
    \underbrace{\Pchi(\infty)}_{=1} - \Pchi[(\mu/\sigma)^2]
}^{\text{contribution
  for} \lambda \geq (\mu/\sigma)^2}
  = 1 + \Pchi[(\mu/\sigma)^2]
  = 1 + \nu
     \fcomma
\end{equation*}
and the conditional probability in \Eq{eq:p:plus:conditional} is given
by
%
\begin{equation}
  \label{eq:p:plus:conditional:only}
  \Prob(\lambda < l | x \geq 0) =
  \begin{cases}
    \displaystyle
    \frac{2 \cdot \Pchi(l)}{1+\nu} & \text{if } l < (\mu/\sigma)^2
    \\
    \displaystyle
    \frac{\Pchi(l) + \nu}{1 + \nu} & \text{otherwise.}
  \end{cases}
\end{equation}
%

\paragraph{$\maybebm{P_-}$:}
For $x<0$, $\lambda$ is a linear function of $x$ and
%
\begin{equation*}
  \lambda < l
  \quad \Leftrightarrow \quad
  x > \xi = \frac{\sigma^2}{2\mu} \left( \frac{\mu^2}{\sigma^2} - l \right)
  \fstop
\end{equation*}
%
Hence, we find for the probability contribution $P_-$
%
\begin{align}
  P_-(l)
  & =
  \Pr(\lambda < l \wedge x < 0; \mu)
  \nonumber
  \\
  & =
  \Pr(\xi < x < 0; \mu)
  \nonumber
  \\
  & =
  \begin{cases}
    \displaystyle
    0 & \text{if } l < (\mu/\sigma)^2 \fcomma \\
    \displaystyle
    \PG(0; \mu, \sigma) - \PG(\xi; \mu, \sigma) & \text{otherwise} \fstop
  \end{cases}
  \label{eq:p:minus:result}
\end{align}
%

\paragraph{}
As numerical approximations for the CDFs of chi-squared and Gaussian
distributions are readily available in many programming libraries,
$P_\lambda$ can be easily computed by combining \Eqs{eq:p:plus:prior},
\eqref{eq:p:plus:conditional:only} and \eqref{eq:p:minus:result} and the
critical $\lambda$-values, $\zeta_\alpha$, can be approximated by, for
example, a bisection algorithm to finally calculate the critical
likelihood-ratio value $c_\alpha = \exp(-2\zeta_\alpha)$.

\printbibliography[heading=bibintoc]

\end{document}


